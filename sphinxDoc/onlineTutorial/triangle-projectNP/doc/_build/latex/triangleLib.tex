%% Generated by Sphinx.
\def\sphinxdocclass{report}
\documentclass[letterpaper,10pt,english]{sphinxmanual}
\ifdefined\pdfpxdimen
   \let\sphinxpxdimen\pdfpxdimen\else\newdimen\sphinxpxdimen
\fi \sphinxpxdimen=.75bp\relax

\usepackage[utf8]{inputenc}
\ifdefined\DeclareUnicodeCharacter
 \ifdefined\DeclareUnicodeCharacterAsOptional
  \DeclareUnicodeCharacter{"00A0}{\nobreakspace}
  \DeclareUnicodeCharacter{"2500}{\sphinxunichar{2500}}
  \DeclareUnicodeCharacter{"2502}{\sphinxunichar{2502}}
  \DeclareUnicodeCharacter{"2514}{\sphinxunichar{2514}}
  \DeclareUnicodeCharacter{"251C}{\sphinxunichar{251C}}
  \DeclareUnicodeCharacter{"2572}{\textbackslash}
 \else
  \DeclareUnicodeCharacter{00A0}{\nobreakspace}
  \DeclareUnicodeCharacter{2500}{\sphinxunichar{2500}}
  \DeclareUnicodeCharacter{2502}{\sphinxunichar{2502}}
  \DeclareUnicodeCharacter{2514}{\sphinxunichar{2514}}
  \DeclareUnicodeCharacter{251C}{\sphinxunichar{251C}}
  \DeclareUnicodeCharacter{2572}{\textbackslash}
 \fi
\fi
\usepackage{cmap}
\usepackage[T1]{fontenc}
\usepackage{amsmath,amssymb,amstext}
\usepackage{babel}
\usepackage{times}
\usepackage[Bjarne]{fncychap}
\usepackage[dontkeepoldnames]{sphinx}

\usepackage{geometry}

% Include hyperref last.
\usepackage{hyperref}
% Fix anchor placement for figures with captions.
\usepackage{hypcap}% it must be loaded after hyperref.
% Set up styles of URL: it should be placed after hyperref.
\urlstyle{same}

\addto\captionsenglish{\renewcommand{\figurename}{Fig.}}
\addto\captionsenglish{\renewcommand{\tablename}{Table}}
\addto\captionsenglish{\renewcommand{\literalblockname}{Listing}}

\addto\captionsenglish{\renewcommand{\literalblockcontinuedname}{continued from previous page}}
\addto\captionsenglish{\renewcommand{\literalblockcontinuesname}{continues on next page}}

\addto\extrasenglish{\def\pageautorefname{page}}

\setcounter{tocdepth}{2}



\title{triangleLib Documentation}
\date{May 30, 2018}
\release{1.0}
\author{Uwe Hernandez Acosta}
\newcommand{\sphinxlogo}{\vbox{}}
\renewcommand{\releasename}{Release}
\makeindex

\begin{document}

\maketitle
\sphinxtableofcontents
\phantomsection\label{\detokenize{index::doc}}



\chapter{Example: guide.rst — The trianglelib guide}
\label{\detokenize{guide:welcome-to-trianglelib-s-documentation}}\label{\detokenize{guide::doc}}\label{\detokenize{guide:example-guide-rst-the-trianglelib-guide}}
Whether you need to test the properties of triangles,
or learn their dimensions, \sphinxcode{trianglelib} does it all!


\section{Special triangles}
\label{\detokenize{guide:special-triangles}}
There are two special kinds of triangle
for which \sphinxcode{trianglelib} offers special support.
\begin{description}
\item[{\sphinxstyleemphasis{Equilateral triangle}}] \leavevmode
All three sides are of equal length.

\item[{\sphinxstyleemphasis{Isosceles triangle}}] \leavevmode
Has at least two sides that are of equal length.

\end{description}

These are supported both by simple methods
that are available in the {\hyperref[\detokenize{api:module-trianglelib.utils}]{\sphinxcrossref{\sphinxcode{trianglelib.utils}}}} module,
and also by a pair of methods of the main
{\hyperref[\detokenize{api:trianglelib.shape.Triangle}]{\sphinxcrossref{\sphinxcode{Triangle}}}} class itself.


\section{Triangle dimensions}
\label{\detokenize{guide:triangle-dimensions}}\label{\detokenize{guide:id1}}
The library can compute triangle perimeter, area,
and can also compare two triangles for equality.
Note that it does not matter which side you start with,
so long as two triangles have the same three sides in the same order!

\begin{sphinxVerbatim}[commandchars=\\\{\}]
\PYG{g+gp}{\PYGZgt{}\PYGZgt{}\PYGZgt{} }\PYG{k+kn}{from} \PYG{n+nn}{trianglelib}\PYG{n+nn}{.}\PYG{n+nn}{shape} \PYG{k}{import} \PYG{n}{Triangle}
\PYG{g+gp}{\PYGZgt{}\PYGZgt{}\PYGZgt{} }\PYG{n}{t1} \PYG{o}{=} \PYG{n}{Triangle}\PYG{p}{(}\PYG{l+m+mi}{3}\PYG{p}{,} \PYG{l+m+mi}{4}\PYG{p}{,} \PYG{l+m+mi}{5}\PYG{p}{)}
\PYG{g+gp}{\PYGZgt{}\PYGZgt{}\PYGZgt{} }\PYG{n}{t2} \PYG{o}{=} \PYG{n}{Triangle}\PYG{p}{(}\PYG{l+m+mi}{4}\PYG{p}{,} \PYG{l+m+mi}{5}\PYG{p}{,} \PYG{l+m+mi}{3}\PYG{p}{)}
\PYG{g+gp}{\PYGZgt{}\PYGZgt{}\PYGZgt{} }\PYG{n}{t3} \PYG{o}{=} \PYG{n}{Triangle}\PYG{p}{(}\PYG{l+m+mi}{3}\PYG{p}{,} \PYG{l+m+mi}{4}\PYG{p}{,} \PYG{l+m+mi}{6}\PYG{p}{)}
\PYG{g+gp}{\PYGZgt{}\PYGZgt{}\PYGZgt{} }\PYG{n+nb}{print} \PYG{n}{t1} \PYG{o}{==} \PYG{n}{t2}
\PYG{g+go}{True}
\PYG{g+gp}{\PYGZgt{}\PYGZgt{}\PYGZgt{} }\PYG{n+nb}{print} \PYG{n}{t1} \PYG{o}{==} \PYG{n}{t3}
\PYG{g+go}{False}
\PYG{g+gp}{\PYGZgt{}\PYGZgt{}\PYGZgt{} }\PYG{n+nb}{print} \PYG{n}{t1}\PYG{o}{.}\PYG{n}{area}\PYG{p}{(}\PYG{p}{)}
\PYG{g+go}{6.0}
\PYG{g+gp}{\PYGZgt{}\PYGZgt{}\PYGZgt{} }\PYG{n+nb}{print} \PYG{n}{t1}\PYG{o}{.}\PYG{n}{scale}\PYG{p}{(}\PYG{l+m+mf}{2.0}\PYG{p}{)}\PYG{o}{.}\PYG{n}{area}\PYG{p}{(}\PYG{p}{)}
\PYG{g+go}{24.0}
\end{sphinxVerbatim}


\section{Valid triangles}
\label{\detokenize{guide:valid-triangles}}
Many combinations of three numbers cannot be the sides of a triangle.
Even if all three numbers are positive instead of negative or zero,
one of the numbers can still be so large
that the shorter two sides
could not actually meet to make a closed figure.
If \(c\) is the longest side, then a triangle is only possible if:
\begin{equation*}
\begin{split}a + b > c
\partial_x \psi (x)\end{split}
\end{equation*}
While the documentation
for each function in the {\hyperref[\detokenize{api:module-trianglelib.utils}]{\sphinxcrossref{\sphinxcode{utils}}}} module
simply specifies a return value for cases that are not real triangles,
the {\hyperref[\detokenize{api:trianglelib.shape.Triangle}]{\sphinxcrossref{\sphinxcode{Triangle}}}} class is more strict
and raises an exception if your sides lengths are not appropriate:

\begin{sphinxVerbatim}[commandchars=\\\{\}]
\PYG{g+gp}{\PYGZgt{}\PYGZgt{}\PYGZgt{} }\PYG{k+kn}{from} \PYG{n+nn}{trianglelib}\PYG{n+nn}{.}\PYG{n+nn}{shape} \PYG{k}{import} \PYG{n}{Triangle}
\PYG{g+gp}{\PYGZgt{}\PYGZgt{}\PYGZgt{} }\PYG{n}{Triangle}\PYG{p}{(}\PYG{l+m+mi}{1}\PYG{p}{,} \PYG{l+m+mi}{1}\PYG{p}{,} \PYG{l+m+mi}{3}\PYG{p}{)}
\PYG{g+gt}{Traceback (most recent call last):}
  \PYG{c}{...}
\PYG{g+gr}{ValueError}: \PYG{n}{one side is too long to make a triangle}
\end{sphinxVerbatim}

If you are not sanitizing your user input
to verify that the three side lengths they are giving you are safe,
then be prepared to trap this exception
and report the error to your user.


\section{The subMod}
\label{\detokenize{guide:the-submod}}\label{\detokenize{guide:module-trianglelib.subMod}}\index{trianglelib.subMod (module)}
This sub module is for testing the includings of several sub docs


\chapter{The triangleLib API Reference}
\label{\detokenize{api::doc}}\label{\detokenize{api:the-trianglelib-api-reference}}

\section{The “shape” module}
\label{\detokenize{api:module-trianglelib.shape}}\label{\detokenize{api:the-shape-module}}\index{trianglelib.shape (module)}\index{Triangle (class in trianglelib.shape)}

\begin{fulllineitems}
\phantomsection\label{\detokenize{api:trianglelib.shape.Triangle}}\pysiglinewithargsret{\sphinxbfcode{class }\sphinxcode{trianglelib.shape.}\sphinxbfcode{Triangle}}{\emph{a}, \emph{b}, \emph{c}}{}
A triangle is a three-sided polygon.
\paragraph{Methods}

\textless{}Here is the description of instantiation.\textgreater{}
\index{is\_similar() (trianglelib.shape.Triangle method)}

\begin{fulllineitems}
\phantomsection\label{\detokenize{api:trianglelib.shape.Triangle.is_similar}}\pysiglinewithargsret{\sphinxbfcode{is\_similar}}{\emph{triangle}}{}
Return whether this triangle is similar to another triangle.

\end{fulllineitems}

\index{is\_equilateral() (trianglelib.shape.Triangle method)}

\begin{fulllineitems}
\phantomsection\label{\detokenize{api:trianglelib.shape.Triangle.is_equilateral}}\pysiglinewithargsret{\sphinxbfcode{is\_equilateral}}{}{}
Return whether this triangle is equilateral.

\end{fulllineitems}

\index{is\_isosceles() (trianglelib.shape.Triangle method)}

\begin{fulllineitems}
\phantomsection\label{\detokenize{api:trianglelib.shape.Triangle.is_isosceles}}\pysiglinewithargsret{\sphinxbfcode{is\_isosceles}}{}{}
Return whether this triangle is isoceles.

\end{fulllineitems}

\index{perimeter() (trianglelib.shape.Triangle method)}

\begin{fulllineitems}
\phantomsection\label{\detokenize{api:trianglelib.shape.Triangle.perimeter}}\pysiglinewithargsret{\sphinxbfcode{perimeter}}{}{}
Return the perimeter of this triangle.

\end{fulllineitems}

\index{area() (trianglelib.shape.Triangle method)}

\begin{fulllineitems}
\phantomsection\label{\detokenize{api:trianglelib.shape.Triangle.area}}\pysiglinewithargsret{\sphinxbfcode{area}}{}{}
Return the area of this triangle.

\end{fulllineitems}

\index{scale() (trianglelib.shape.Triangle method)}

\begin{fulllineitems}
\phantomsection\label{\detokenize{api:trianglelib.shape.Triangle.scale}}\pysiglinewithargsret{\sphinxbfcode{scale}}{\emph{factor}}{}
Return a new triangle, \sphinxtitleref{factor} times the size of this one.

\end{fulllineitems}


\end{fulllineitems}



\section{The “utils” module}
\label{\detokenize{api:module-trianglelib.utils}}\label{\detokenize{api:the-utils-module}}\index{trianglelib.utils (module)}
Routines to test triangle properties without explicit instantiation.
\index{compute\_area() (in module trianglelib.utils)}

\begin{fulllineitems}
\phantomsection\label{\detokenize{api:trianglelib.utils.compute_area}}\pysiglinewithargsret{\sphinxcode{trianglelib.utils.}\sphinxbfcode{compute\_area}}{\emph{a}, \emph{b}, \emph{c}}{}
Return the area of the triangle with side lengths \sphinxtitleref{a}, \sphinxtitleref{b}, and \sphinxtitleref{c}.

If the three lengths provided cannot be the sides of a triangle,
then the area 0 is returned.

\end{fulllineitems}

\index{compute\_perimeter() (in module trianglelib.utils)}

\begin{fulllineitems}
\phantomsection\label{\detokenize{api:trianglelib.utils.compute_perimeter}}\pysiglinewithargsret{\sphinxcode{trianglelib.utils.}\sphinxbfcode{compute\_perimeter}}{\emph{a}, \emph{b}, \emph{c}}{}
Return the perimeer of the triangle with side lengths \sphinxtitleref{a}, \sphinxtitleref{b}, and \sphinxtitleref{c}.

If the three lengths provided cannot be the sides of a triangle,
then the perimeter 0 is returned.

\end{fulllineitems}

\index{is\_equilateral() (in module trianglelib.utils)}

\begin{fulllineitems}
\phantomsection\label{\detokenize{api:trianglelib.utils.is_equilateral}}\pysiglinewithargsret{\sphinxcode{trianglelib.utils.}\sphinxbfcode{is\_equilateral}}{\emph{a}, \emph{b}, \emph{c}}{}
Return whether lengths \sphinxtitleref{a}, \sphinxtitleref{b}, and \sphinxtitleref{c} are an equilateral triangle.

\end{fulllineitems}

\index{is\_isosceles() (in module trianglelib.utils)}

\begin{fulllineitems}
\phantomsection\label{\detokenize{api:trianglelib.utils.is_isosceles}}\pysiglinewithargsret{\sphinxcode{trianglelib.utils.}\sphinxbfcode{is\_isosceles}}{\emph{a}, \emph{b}, \emph{c}}{}
Return whether lengths \sphinxtitleref{a}, \sphinxtitleref{b}, and \sphinxtitleref{c} are an isosceles triangle.

\end{fulllineitems}

\index{is\_triangle() (in module trianglelib.utils)}

\begin{fulllineitems}
\phantomsection\label{\detokenize{api:trianglelib.utils.is_triangle}}\pysiglinewithargsret{\sphinxcode{trianglelib.utils.}\sphinxbfcode{is\_triangle}}{\emph{a}, \emph{b}, \emph{c}}{}
Return whether lengths \sphinxtitleref{a}, \sphinxtitleref{b}, \sphinxtitleref{c} can be the sides of a triangle.

\end{fulllineitems}



\chapter{The “subMod” module}
\label{\detokenize{subMod:the-submod-module}}\label{\detokenize{subMod::doc}}\label{\detokenize{subMod:module-trianglelib.subMod}}\index{trianglelib.subMod (module)}
This sub module is for testing the includings of several sub docs


\section{The ‘subFunc’ Module}
\label{\detokenize{subMod:module-trianglelib.subMod.subFunc}}\label{\detokenize{subMod:the-subfunc-module}}\index{trianglelib.subMod.subFunc (module)}
This is the docstring for the example.py module.  Modules names should
have short, all-lowercase names.  The module name may have underscores if
this improves readability.

Every module should have a docstring at the very top of the file.  The
module’s docstring may extend over multiple lines.  If your docstring does
extend over multiple lines, the closing three quotation marks must be on
a line by itself, preferably preceded by a blank line.
\index{foo() (in module trianglelib.subMod.subFunc)}

\begin{fulllineitems}
\phantomsection\label{\detokenize{subMod:trianglelib.subMod.subFunc.foo}}\pysiglinewithargsret{\sphinxcode{trianglelib.subMod.subFunc.}\sphinxbfcode{foo}}{\emph{var1}, \emph{var2}, \emph{long\_var\_name='hi'}}{}
A one-line summary that does not use variable names or the
function name.

Several sentences providing an extended description. Refer to
variables using back-ticks, e.g. \sphinxtitleref{var}.
\begin{quote}\begin{description}
\item[{Parameters}] \leavevmode
\sphinxstylestrong{var1} : array\_like
\begin{quote}

Array\_like means all those objects \textendash{} lists, nested lists, etc. \textendash{}
that can be converted to an array.  We can also refer to
variables like \sphinxtitleref{var1}.
\end{quote}

\sphinxstylestrong{var2} : int
\begin{quote}

The type above can either refer to an actual Python type
(e.g. \sphinxcode{int}), or describe the type of the variable in more
detail, e.g. \sphinxcode{(N,) ndarray} or \sphinxcode{array\_like}.
\end{quote}

\sphinxstylestrong{long\_var\_name} : \{‘hi’, ‘ho’\}, optional
\begin{quote}

Choices in brackets, default first when optional.
\end{quote}

\item[{Returns}] \leavevmode
type
\begin{quote}

Explanation of anonymous return value of type \sphinxcode{type}.
\end{quote}

\sphinxstylestrong{describe} : type
\begin{quote}

Explanation of return value named \sphinxtitleref{describe}.
\end{quote}

\sphinxstylestrong{out} : type
\begin{quote}

Explanation of \sphinxtitleref{out}.
\end{quote}

type\_without\_description

\item[{Other Parameters}] \leavevmode
\sphinxstylestrong{only\_seldom\_used\_keywords} : type
\begin{quote}

Explanation
\end{quote}

\sphinxstylestrong{common\_parameters\_listed\_above} : type
\begin{quote}

Explanation
\end{quote}

\item[{Raises}] \leavevmode
\sphinxstylestrong{BadException}
\begin{quote}

Because you shouldn’t have done that.
\end{quote}

\end{description}\end{quote}


\sphinxstrong{See also:}

\begin{description}
\item[{\sphinxcode{otherfunc}}] \leavevmode
relationship (optional)

\item[{\sphinxcode{newfunc}}] \leavevmode
Relationship (optional), which could be fairly long, in which case the line wraps here.

\end{description}

\sphinxcode{thirdfunc}, \sphinxcode{fourthfunc}, \sphinxcode{fifthfunc}


\paragraph{Notes}

Notes about the implementation algorithm (if needed).

This can have multiple paragraphs.

You may include some math:
\begin{equation*}
\begin{split}X(e^{j\omega } ) = x(n)e^{ - j\omega n}\end{split}
\end{equation*}
And even use a greek symbol like \(omega\) inline.
\paragraph{References}

Cite the relevant literature, e.g. \phantomsection\label{\detokenize{subMod:id1}}{\hyperref[\detokenize{subMod:r11}]{\sphinxcrossref{{[}R11{]}}}}.  You may also cite these
references in the notes section above.

\phantomsection\label{\detokenize{subMod:id2}}{\hyperref[\detokenize{subMod:r11}]{\sphinxcrossref{{[}R11{]}}}}
\paragraph{Examples}

These are written in doctest format, and should illustrate how to
use the function.

\begin{sphinxVerbatim}[commandchars=\\\{\}]
\PYG{g+gp}{\PYGZgt{}\PYGZgt{}\PYGZgt{} }\PYG{n}{a} \PYG{o}{=} \PYG{p}{[}\PYG{l+m+mi}{1}\PYG{p}{,} \PYG{l+m+mi}{2}\PYG{p}{,} \PYG{l+m+mi}{3}\PYG{p}{]}
\PYG{g+gp}{\PYGZgt{}\PYGZgt{}\PYGZgt{} }\PYG{n+nb}{print} \PYG{p}{[}\PYG{n}{x} \PYG{o}{+} \PYG{l+m+mi}{3} \PYG{k}{for} \PYG{n}{x} \PYG{o+ow}{in} \PYG{n}{a}\PYG{p}{]}
\PYG{g+go}{[4, 5, 6]}
\PYG{g+gp}{\PYGZgt{}\PYGZgt{}\PYGZgt{} }\PYG{n+nb}{print} \PYG{l+s+s2}{\PYGZdq{}}\PYG{l+s+s2}{a}\PYG{l+s+se}{\PYGZbs{}n}\PYG{l+s+se}{\PYGZbs{}n}\PYG{l+s+s2}{b}\PYG{l+s+s2}{\PYGZdq{}}
\PYG{g+go}{a}
\PYG{g+go}{b}
\end{sphinxVerbatim}

\end{fulllineitems}


\begin{sphinxthebibliography}{R11}
\bibitem[R11]{\detokenize{R11}}{\phantomsection\label{\detokenize{subMod:r11}} 
O. McNoleg, “The integration of GIS, remote sensing,
expert systems and adaptive co-kriging for environmental habitat
modelling of the Highland Haggis using object-oriented, fuzzy-logic
and neural-network techniques,” Computers \& Geosciences, vol. 22,
pp. 585-588, 1996.
}
\end{sphinxthebibliography}


\renewcommand{\indexname}{Python Module Index}
\begin{sphinxtheindex}
\def\bigletter#1{{\Large\sffamily#1}\nopagebreak\vspace{1mm}}
\bigletter{t}
\item {\sphinxstyleindexentry{trianglelib.shape}}\sphinxstyleindexpageref{api:\detokenize{module-trianglelib.shape}}
\item {\sphinxstyleindexentry{trianglelib.subMod}}\sphinxstyleindexpageref{subMod:\detokenize{module-trianglelib.subMod}}
\item {\sphinxstyleindexentry{trianglelib.subMod.subFunc}}\sphinxstyleindexpageref{subMod:\detokenize{module-trianglelib.subMod.subFunc}}
\item {\sphinxstyleindexentry{trianglelib.utils}}\sphinxstyleindexpageref{api:\detokenize{module-trianglelib.utils}}
\end{sphinxtheindex}

\renewcommand{\indexname}{Index}
\printindex
\end{document}