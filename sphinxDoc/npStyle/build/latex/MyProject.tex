%% Generated by Sphinx.
\def\sphinxdocclass{report}
\documentclass[letterpaper,10pt,english]{sphinxmanual}
\ifdefined\pdfpxdimen
   \let\sphinxpxdimen\pdfpxdimen\else\newdimen\sphinxpxdimen
\fi \sphinxpxdimen=.75bp\relax

\PassOptionsToPackage{warn}{textcomp}
\usepackage[utf8]{inputenc}
\ifdefined\DeclareUnicodeCharacter
 \ifdefined\DeclareUnicodeCharacterAsOptional
  \DeclareUnicodeCharacter{"00A0}{\nobreakspace}
  \DeclareUnicodeCharacter{"2500}{\sphinxunichar{2500}}
  \DeclareUnicodeCharacter{"2502}{\sphinxunichar{2502}}
  \DeclareUnicodeCharacter{"2514}{\sphinxunichar{2514}}
  \DeclareUnicodeCharacter{"251C}{\sphinxunichar{251C}}
  \DeclareUnicodeCharacter{"2572}{\textbackslash}
 \else
  \DeclareUnicodeCharacter{00A0}{\nobreakspace}
  \DeclareUnicodeCharacter{2500}{\sphinxunichar{2500}}
  \DeclareUnicodeCharacter{2502}{\sphinxunichar{2502}}
  \DeclareUnicodeCharacter{2514}{\sphinxunichar{2514}}
  \DeclareUnicodeCharacter{251C}{\sphinxunichar{251C}}
  \DeclareUnicodeCharacter{2572}{\textbackslash}
 \fi
\fi
\usepackage{cmap}
\usepackage[T1]{fontenc}
\usepackage{amsmath,amssymb,amstext}
\usepackage{babel}
\usepackage{times}
\usepackage[Bjarne]{fncychap}
\usepackage{sphinx}

\usepackage{geometry}

% Include hyperref last.
\usepackage{hyperref}
% Fix anchor placement for figures with captions.
\usepackage{hypcap}% it must be loaded after hyperref.
% Set up styles of URL: it should be placed after hyperref.
\urlstyle{same}
\addto\captionsenglish{\renewcommand{\contentsname}{Contents:}}

\addto\captionsenglish{\renewcommand{\figurename}{Fig.}}
\addto\captionsenglish{\renewcommand{\tablename}{Table}}
\addto\captionsenglish{\renewcommand{\literalblockname}{Listing}}

\addto\captionsenglish{\renewcommand{\literalblockcontinuedname}{continued from previous page}}
\addto\captionsenglish{\renewcommand{\literalblockcontinuesname}{continues on next page}}

\addto\extrasenglish{\def\pageautorefname{page}}

\setcounter{tocdepth}{1}



\title{MyProject Documentation}
\date{May 08, 2018}
\release{0.0.1}
\author{Uwe Hernandez Acosta}
\newcommand{\sphinxlogo}{\vbox{}}
\renewcommand{\releasename}{Release}
\makeindex

\begin{document}

\maketitle
\sphinxtableofcontents
\phantomsection\label{\detokenize{example::doc}}



\chapter{Example}
\label{\detokenize{example:example}}\label{\detokenize{example:welcome-to-myproject-s-documentation}}\label{\detokenize{example:module-example.example}}\index{example.example (module)}
This is the docstring for the example.py module.  Modules names should
have short, all-lowercase names.  The module name may have underscores if
this improves readability.

Every module should have a docstring at the very top of the file.  The
module’s docstring may extend over multiple lines.  If your docstring does
extend over multiple lines, the closing three quotation marks must be on
a line by itself, preferably preceded by a blank line.
\index{foo() (in module example.example)}

\begin{fulllineitems}
\phantomsection\label{\detokenize{example:example.example.foo}}\pysiglinewithargsret{\sphinxcode{\sphinxupquote{example.example.}}\sphinxbfcode{\sphinxupquote{foo}}}{\emph{var1}, \emph{var2}, \emph{long\_var\_name='hi'}}{}
A one-line summary that does not use variable names or the
function name.

Several sentences providing an extended description. Refer to
variables using back-ticks, e.g. \sphinxtitleref{var}.
\begin{quote}\begin{description}
\item[{Parameters}] \leavevmode\begin{description}
\item[{\sphinxstylestrong{var1}}] \leavevmode{[}array\_like{]}
Array\_like means all those objects \textendash{} lists, nested lists, etc. \textendash{}
that can be converted to an array.  We can also refer to
variables like \sphinxtitleref{var1}.

\item[{\sphinxstylestrong{var2}}] \leavevmode{[}int{]}
The type above can either refer to an actual Python type
(e.g. \sphinxcode{\sphinxupquote{int}}), or describe the type of the variable in more
detail, e.g. \sphinxcode{\sphinxupquote{(N,) ndarray}} or \sphinxcode{\sphinxupquote{array\_like}}.

\item[{\sphinxstylestrong{long\_var\_name}}] \leavevmode{[}\{‘hi’, ‘ho’\}, optional{]}
Choices in brackets, default first when optional.

\end{description}

\item[{Returns}] \leavevmode\begin{description}
\item[{\sphinxstylestrong{type}}] \leavevmode
Explanation of anonymous return value of type \sphinxcode{\sphinxupquote{type}}.

\item[{\sphinxstylestrong{describe}}] \leavevmode{[}type{]}
Explanation of return value named \sphinxtitleref{describe}.

\item[{\sphinxstylestrong{out}}] \leavevmode{[}type{]}
Explanation of \sphinxtitleref{out}.

\item[{\sphinxstylestrong{type\_without\_description}}] \leavevmode
\end{description}

\item[{Other Parameters}] \leavevmode\begin{description}
\item[{\sphinxstylestrong{only\_seldom\_used\_keywords}}] \leavevmode{[}type{]}
Explanation

\item[{\sphinxstylestrong{common\_parameters\_listed\_above}}] \leavevmode{[}type{]}
Explanation

\end{description}

\item[{Raises}] \leavevmode\begin{description}
\item[{\sphinxstylestrong{BadException}}] \leavevmode
Because you shouldn’t have done that.

\end{description}

\end{description}\end{quote}


\sphinxstrong{See also:}

\begin{description}
\item[{\sphinxcode{\sphinxupquote{otherfunc}}}] \leavevmode
relationship (optional)

\item[{\sphinxcode{\sphinxupquote{newfunc}}}] \leavevmode
Relationship (optional), which could be fairly long, in which case the line wraps here.

\end{description}

\sphinxcode{\sphinxupquote{thirdfunc}}, \sphinxcode{\sphinxupquote{fourthfunc}}, \sphinxcode{\sphinxupquote{fifthfunc}}


\paragraph{Notes}

Notes about the implementation algorithm (if needed).

This can have multiple paragraphs.

You may include some math:
\begin{equation*}
\begin{split}X(e^{j\omega } ) = x(n)e^{ - j\omega n}\end{split}
\end{equation*}
And even use a greek symbol like \(omega\) inline.
\paragraph{References}

Cite the relevant literature, e.g. \phantomsection\label{\detokenize{example:id1}}{\hyperref[\detokenize{example:rfb95ad3cf4e1-1}]{\sphinxcrossref{{[}1{]}}}}.  You may also cite these
references in the notes section above.

\phantomsection\label{\detokenize{example:id2}}{\hyperref[\detokenize{example:rfb95ad3cf4e1-1}]{\sphinxcrossref{{[}1{]}}}}
\paragraph{Examples}

These are written in doctest format, and should illustrate how to
use the function.

\fvset{hllines={, ,}}%
\begin{sphinxVerbatim}[commandchars=\\\{\}]
\PYG{g+gp}{\PYGZgt{}\PYGZgt{}\PYGZgt{} }\PYG{n}{a} \PYG{o}{=} \PYG{p}{[}\PYG{l+m+mi}{1}\PYG{p}{,} \PYG{l+m+mi}{2}\PYG{p}{,} \PYG{l+m+mi}{3}\PYG{p}{]}
\PYG{g+gp}{\PYGZgt{}\PYGZgt{}\PYGZgt{} }\PYG{n+nb}{print} \PYG{p}{[}\PYG{n}{x} \PYG{o}{+} \PYG{l+m+mi}{3} \PYG{k}{for} \PYG{n}{x} \PYG{o+ow}{in} \PYG{n}{a}\PYG{p}{]}
\PYG{g+go}{[4, 5, 6]}
\PYG{g+gp}{\PYGZgt{}\PYGZgt{}\PYGZgt{} }\PYG{n+nb}{print} \PYG{l+s+s2}{\PYGZdq{}}\PYG{l+s+s2}{a}\PYG{l+s+se}{\PYGZbs{}n}\PYG{l+s+se}{\PYGZbs{}n}\PYG{l+s+s2}{b}\PYG{l+s+s2}{\PYGZdq{}}
\PYG{g+go}{a}
\PYG{g+go}{b}
\end{sphinxVerbatim}

\end{fulllineitems}



\chapter{subModule}
\label{\detokenize{example:submodule}}\label{\detokenize{example:module-subMod.test}}\index{subMod.test (module)}
a random function

used to test the subMod documentation
\index{func() (in module subMod.test)}

\begin{fulllineitems}
\phantomsection\label{\detokenize{example:subMod.test.func}}\pysiglinewithargsret{\sphinxcode{\sphinxupquote{subMod.test.}}\sphinxbfcode{\sphinxupquote{func}}}{\emph{arg1}, \emph{arg2}}{}
Summary line.

Extended description of function.
\begin{quote}\begin{description}
\item[{Parameters}] \leavevmode\begin{description}
\item[{\sphinxstylestrong{arg1}}] \leavevmode{[}int{]}
Description of arg1

\item[{\sphinxstylestrong{arg2}}] \leavevmode{[}str{]}
Description of arg2

\end{description}

\item[{Returns}] \leavevmode\begin{description}
\item[{\sphinxstylestrong{bool}}] \leavevmode
Description of return value

\end{description}

\end{description}\end{quote}

\end{fulllineitems}



\chapter{Indices and tables}
\label{\detokenize{example:indices-and-tables}}\begin{itemize}
\item {} 
\DUrole{xref,std,std-ref}{genindex}

\item {} 
\DUrole{xref,std,std-ref}{modindex}

\item {} 
\DUrole{xref,std,std-ref}{search}

\end{itemize}

\begin{sphinxthebibliography}{1}
\bibitem[1]{\detokenize{1}}{\phantomsection\label{\detokenize{example:rfb95ad3cf4e1-1}} 
O. McNoleg, “The integration of GIS, remote sensing,
expert systems and adaptive co-kriging for environmental habitat
modelling of the Highland Haggis using object-oriented, fuzzy-logic
and neural-network techniques,” Computers \& Geosciences, vol. 22,
pp. 585-588, 1996.
}
\end{sphinxthebibliography}


\renewcommand{\indexname}{Python Module Index}
\begin{sphinxtheindex}
\def\bigletter#1{{\Large\sffamily#1}\nopagebreak\vspace{1mm}}
\bigletter{e}
\item {\sphinxstyleindexentry{example.example}}\sphinxstyleindexpageref{example:\detokenize{module-example.example}}
\indexspace
\bigletter{s}
\item {\sphinxstyleindexentry{subMod.test}}\sphinxstyleindexpageref{example:\detokenize{module-subMod.test}}
\end{sphinxtheindex}

\renewcommand{\indexname}{Index}
\printindex
\end{document}